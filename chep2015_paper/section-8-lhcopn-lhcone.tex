In September/October 2014 at the WLCG Grid Deployment Board, the two LHC General Purpose Experiments  requested Tier-1s to join the HEPiX-IPv6 working group and further encouraged sites to move their production endpoints to dual stack even if this resulted in a reduction of their site reliability and site availability. The proposal of the LHC experiment Atlas was to
\begin{itemize}
 \item request that all Tier-1s provide,
	besides an IPv6 peering to LHCOPN,
        a dual stack PerfSONAR machine by April 2015
 \item request that Tier-2s provide,
        besides an IPv6 peering to their LHCONE connection,
        a dual stack PerfSONAR machine by August 2015.
\end{itemize}
At the last LHC[OPN/ONE] meeting in February this year a proposal containing the IPv6 request was put forward. There were no objections from the site representatives nor from the NRENs to this proposal. The following Tier-1 sites are actively announcing an IPv6 peering to LHCOPN: CH-CERN, DE-KIT, ES-PIC, FR-CCIN2P3, NDGF, NL-T1. IT-INFN-CNAF is in the process of preparing the IPv6 peering. The group of IPv6 peers over LHCONE is currently even smaller: besides CH-CERN this are the two sites CEA SACLAY and FR-CCIN2P3.
The ipv6 peerings are reflected at the PerfSONAR dualstack dashboard url:\\{\tt\small http://maddash.aglt2.org/maddash-webui/index.cgi?dashboard=Dual-Stack\%20Mesh\%20Config} \\ It implies that there are still some LHC tier-1 sites, more than a month behind the schedule agreed upon, not announcing their ipv6 cidr to LHCONE.


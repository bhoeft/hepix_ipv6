\textbf{ALICE}
The legacy Grid services of ALICE (AliEn [cit]) are built around Perl 5.10 and Xrootd 3.0 that don’t support IPv6 natively. The new set of services (jAliEn [cit]) are written in Java and using the latest Xrootd for data transfers, both fully supporting IPv6. Central services run both software stacks with functionality being migrated in steps from the legacy to the new version. One important step was the migration of data transfer tools to the new version, in particular all RAW data and calibration files are exported from Tier 0 to Tier 1s with the new tools. In progress is the migration of the experiment software to ROOT6 and a recent Xrootd library for data access, allowing job deployment on IPv6 resources.
Because of the fully federated storage under the experiment computing model a ~full IPv6 storage deployment is required before IPv6-only worker nodes can run jobs. At the moment of writing this article a 54% of the storage volume is dual stacked with less than 5% of the volume running old server software versions that are not IPv6-ready.

\textbf{ATLAS}


\textbf{CMS}
The submission infrastructure in CMS is based on glideinWMS [cit] and HTCondor [cit], which are fully IPv6-compliant. The deployment of IPv6 on the relevant service nodes has been completed at CERN but not at FNAL, where it is scheduled by the end of the year. It is expected that, after a period of testing in the Integration Testbed (ITB), the production infrastructure will see IPv6 enabled across the board.

\textbf{LHCb}
The LHCb infrastructure is built around DIRAC [cit] which is fully IPv6-compliant and the servers are based at CERN. All the servers running DIRAC services are dual-stacked and happy to listen to connections on either IPv4 or IPv6 as they come. The LHCb experiment is now waiting for the sites to deploy the pledged storages in dual-stack mode. With the storages already available (38% of LHCb Tier-2D by volume and X% Tier-0/1 by volume) of  and the automatic failover mechanisms in place, LHCb is already running on IPv4 or IPv6 or dual-stacked worker nodes without any manual intervention.

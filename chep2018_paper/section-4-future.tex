%section 4
\subsection{Obstacles and potential show-stoppers}
Our experience with the transition of WLCG Tier-1 and Tier-2 centers so far 
has identified various cases where the IPv4$\rightarrow$IPv6 transition has
consequences that exceed the simple replacement of the IP {\it transport} layer.
These broadly fall in the following categories:
\begin{enumerate}
\item Software components and protocol with fixed-size storage for network
addresses, for example in (Grid-)FTP and AFS. This can be overcome by appropriate
protocol extensions (e.g. the introduction of 'extended' FTP commands
{\tt EPRT}, {\tt EPSV}, etc.), with a large development effort required.
In certain cases (AFS) this effort was determined to be too large and not
worthwhile, while the GrifFTP 'v2' extensions were issued with no IPv6 support.
\item Software components and protocols that assume single addresses or a single
IP protocol for 
network endpoints (to various extents, all the components in the WLCG software
matrix). While all Operating Systems do provide hybrid network stacks and prioritized/configurable
source and destination address selection\footnote{Address selection is regulated in most cases by RFC6724.}, applications should always iterate over multiple
possible results (belonging to multiple IP protocol versions) and provide configurable
overrides and preferences. 
\item Software components and network infrastructures providing asymmetric
handling, or separate code stacks, for IPv4 and IPv6 traffic: these also
have the ability to load the network transport choice with measurable
performance consequences.
\end{enumerate}
Detecting and explaining these IPv4/IPv6 asymmetries and fostering improved
symmetry across the WLCG software matrix has been a long-standing activity
for our group. As there is inherent risk in changing, testing and rolling
out (sometimes sizable) software changes when no functional
issue requires immediate attention, these changes are often not shortlisted
for deployment by software development teams. We need to continue tracking
these issues down to the
level where no deployment of measurable performance asymmetry between the two
protocols is seen: this can be seen as the backdrop of any future activity.

\subsection{Further steps}

\subsection{IPv6-only scenarios}

\subsection{Conclusion}

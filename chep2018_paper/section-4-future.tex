%section 4
\subsection{Obstacles and potential show-stoppers}
Our experience with the transition of WLCG Tier-1 and Tier-2 centers so far 
has identified various cases where the IPv4$\rightarrow$IPv6 transition has
consequences that exceed the simple replacement of the IP {\it transport} layer.
These broadly fall in the following categories:
\begin{enumerate}
\item Software components and protocol with fixed-size storage for network
addresses, for example in (Grid-)FTP and AFS. This can be overcome by appropriate
protocol extensions (e.g. the introduction of 'extended' FTP commands
{\tt EPRT}, {\tt EPSV}, etc.), with a large development effort required.
In certain cases (AFS) this effort was determined to be too large and not
worthwhile, while the GrifFTP 'v2' extensions were issued with no IPv6 support.
\item Software components and protocols that assume single addresses or a single
IP protocol for 
network endpoints (to various extents, all the components in the WLCG software
matrix). While all Operating Systems do provide hybrid network stacks and prioritized/configurable
source and destination address selection\footnote{Address selection is regulated in most cases by RFC6724.}, applications should always iterate over multiple
possible results (belonging to multiple IP protocol versions) and provide configurable
overrides and preferences. 
\item Software components and network infrastructures providing asymmetric
handling, or separate code stacks, for IPv4 and IPv6 traffic: these also
have the ability to load the network transport choice with measurable
performance consequences.
\end{enumerate}
Detecting and explaining these IPv4/IPv6 asymmetries and fostering improved
symmetry across the WLCG software matrix has been a long-standing activity
for our group. As there is inherent risk in changing, testing and rolling
out (sometimes sizable) software changes when no functional
issue requires immediate attention, these changes are often not shortlisted
for deployment by software development teams. We need to continue tracking
these issues down to the
level where no deployment of measurable performance asymmetry between the two
protocols is seen: this can be seen as the backdrop of any future activity.

\subsection{Further steps}
Once the data transfer monitoring infrastructure described in Section
\ref{sec-monitoring} is completed and covers all services, two cases
need to be consistently monitored and, where needed, investigated
and resolved:
\begin{enumerate}
\item cases where the fraction of data transferred over IPv6 is lower than expected:
the preference for IPv6 over IPv4 has to established throughout the system;
\item cases where the transfer performance is either significantly worse 
or significantly better on IPv6 over IPv4: asymmetries in the routing and
transport should be identified, especially in the LHCONE and LHCOPN networks.
\end{enumerate}
\par
Over a stable and understood data transfer network 
IPv6-only worker nodes should encounter no more operational anomalies
than other types of worker nodes. The 
job failure rate on IPV6-only, dual-stack and IPv4-only worker nodes should
be monitored statistically and deviations properly troubleshot.
\par
Once the use-case of IPv6-only worker nodes operates smoothly, the next
step in the transition roadmap is moving services and network segments to
IPv6-only operation.

\subsection{IPv6-only scenarios}
The final step of the IPv4$\rightarrow$IPv6 transition is the phasing out of
IPv4. While still in the non-near future, this is also arguably the only step
that can bring along a desirable reduction
in the infrastructure maintenance effort, so it shouldn't be delayed
without reason. Many of the issues described in the previous sections 
actually arise
from the fact that in the current transition phase many components and
configurations have to be maintained in parallel.
\par
In order to demonstrate that the close of the transition is a realistic
objective, IPv6-only operation of an entire WLCG Tier-X site will
need to have been proven by all experiments and sites run by most WLCG
partners, with residual IPv4-only services either deprecated or made
accessible via protocol translation techniques (e.g. DNS64/NAT64, see
NAT64/DNS64 (RFC6146/6147 \cite{rfc})). 
\par
{\it XXX Should we mention the few experiments performed so far here ?}

\subsection{Conclusion}
The target case of allowing successful execution of generic WLCG job payloads
on IPv6-only worker nodes has been driving the IPv4-IPv6 dual-stack deployment
of storage services across the WLCG production infrastructure. According to
the current progress tracking, summarised in Sections \ref{ssec-t1trans} and
\ref{ssec-t2trans} above, this step should be completed at CERN, Tier-1
and Tier-2 centers by the end of LHC Run-II (end of 2018). A few residual
issues were identified in the software stack in this process.
\par
Data from
a complete and pervasive monitoring infrastructure are crucial in building
confidence in the new transport layer, which is expected to provide the
{\it same} level of performance and reliability as the one it replaces: these
come from various sources (see Section \ref{sec-monitoring}), with a few
remaining blind spots being covered.
\par
Feedback from the actual operation of  
IPv6-only computing farms will then determine the time-scale and viability of
driving the transition process to its natural ending, when the burden
of operating a duplicated transport infrastructure will be eventually lifted.


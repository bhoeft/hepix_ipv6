The impact of IPv6 is not limited to the transport layer 
but introduces the need for choice and preference in name-to-address 
resolution, implies multi-homing of all network endpoints (possibly on 
multiple protocol versions) and requires opaque handling of address 
information. This broadens the scope of code changes needed to add
IPv6 support to existing code and adds to the complexity of testing:
continued operation on IPv4 on dual-stack hosts, then preference of IPv6 and 
options to control it for all network bindings and connections
need to be verified with adequate code coverage.
\par
At opposite ends of the spectrum of practical testing options are
testing of individual, isolated components and services and the analysis of 
integrated services on dual-stack nodes. Both approaches are incomplete:
\begin{itemize}
\item[-] testing of isolated components misses the interaction with other
services at the OS level and usually requires services to be configured
differently than for production;
\item[-] testing of production-ready, integrated nodes may just be 
accidentally focusing on normal operation and bring insufficient 
code and functionality coverage.
\end{itemize}
This calls for a complementary approach,
where individual services are deployed and tested within the scale of
available dedicated resources and, once sufficient confidence and knowledge
of their level and mode of IPv6 support is built, are watched in the context
of a production node with dual-stack network and dual IPv4/IPv6 public address
resolution.
\par
Desirable characteristics of a dedicated testbed for single-service testing are:
\begin{itemize}
\item Geographical spread covering all ranges of realistic network latencies
and as many network providers as possible.
\item Uniform authentication/authorization scheme to factor out AA issues.
\item Uniform OS installation, to factor out any issue with the custom 
configuration needed to test isolated services and for easier
deployment of new services.
\end{itemize}
The current list of active testbed nodes can be found at the following URL:\\
{\tt http://hepix-ipv6.web.cern.ch/testbed-nodes}\\
While we have at the time of writing a reasonable
9-site/6-National Research and Education Network (NREN) coverage of Europe, the only non-european sites in the testbed
are IHEP Beijing in China and Fermilab in the US. More testbed sites are both needed and welcome to join 
to achieve a better match of our stated testbed goals.
\par
As for OS distributions, testbed nodes are mostly installed with 
Scientific Linux (CERN) version 5,
to replicate production conditions at LCG Tier-X centres. A few testbed nodes
have Red Hat Enterprise Linux (RHEL) \cite{rhel} version 6 derivatives installed: this allowed us to discover 
(and document in our knowledge base, {\tt http://hepix-ipv6.web.cern.ch/knowledge-base})
that, rather unexpectedly, {\tt libc} on RHEL version 6 causes unspecified 
protocol sockets to be bound on IPv4 only, instead of dual-stack 
as it used to be.
\par
To achieve the simplest possible authentication scheme, a custom Globus
Security Infrastructure (GSI)
plug-in maps all members of our test Virtual Organisation (VO) ({\tt ipv6.hepix.org}) to one
local account, logging any access. 
\par
The first service we deployed for standalone testing through the testbed was
GridFTP (or, more accurately, {\tt gsiftp}). 
This was not only because of GridFTP's basic role in WLCG data
transfer, but also because the FTP protocol (the GSI extensions don't 
affect but also suffer from this issue) is a paradigmatic example of 
how non-trivial IPv6 support can be. The original FTP specifications 
(RFC 765/RFC 959 \cite{rfc}) used the quad-byte notation for IP addresses 
{\em in the syntax of the FTP protocol commands} {\tt PORT} and {\tt PASV}.
This required the introduction, with RFC 2428 (September 1998) \cite{rfc} of
``extended'' versions of the same commands, supporting different address
families, and IPv6 in particular. We found on our testbed that support for 
the 'extended' command forms (and thus {\em implicitly} for IPv6)
is missing from certain FTP client implementations. Retrofitting the
clients with these commands is definitely more than a simple change in
the transport layer and serves as an example of how ramified the
introduction of 'IPv6 support' can be.
\par
Building on this mesh of GridFTP servers, both continuous direct point-to-point 
file transfer tests and tests of the File Transfer Service 
(FTS), were successfully carried out. Storage Resource Manager (SRM) endpoints
were also added, as described in detail in the next section.

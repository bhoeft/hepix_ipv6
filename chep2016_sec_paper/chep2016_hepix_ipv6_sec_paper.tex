\documentclass[a4paper]{jpconf}
\usepackage{graphicx}
\usepackage{color}
\usepackage{array}
\usepackage{enumerate}
\usepackage{framed}
%\usepackage{url}
%\usepackage{lineno}
%\linenumbers

\begin{document}
\title{IPv6 Security}

\author{M~Babik$^1$, J~Chudoba$^2$, A~Dewhurst$^3$, T~Finnern$^4$, T~Froy$^5$,
        C~Grigoras$^1$, K~Hafeez$^3$, B~Hoeft$^6$, T~Idiculla$^3$, D~P~Kelsey$^3$,  
        F~L\'opez~Mu\~noz$^{7,8}$, E~Martelli$^1$, R~Nandakumar$^3$, 
        K~Ohrenberg$^4$, F~Prelz$^{9}$, D~Rand$^{10}$, 
        A~Sciab\`a$^1$, U~Tigerstedt$^{11}$, D~Traynor$^5$ and R~Wartel$^1$}

%\address{$^1$ IN2P3 Computing Centre, Boulevard du 11 Novembre 1918, F-69622 Villeurbanne Cedex, France}
\address{$^1$ European Organization for Nuclear Research (CERN), CH-1211 Geneva 23, Switzerland}
\address{$^2$ Institute of Physics, Academy of Sciences of the Czech Republic Na Slovance 2 182 21 Prague 8, Czech Republic}
\address{$^3$ STFC Rutherford Appleton Laboratory, Harwell Campus, Didcot, Oxfordshire OX11 0QX, United Kingdom}
\address{$^4$ Deutsches Elektronen-Synchrotron DESY, Notkestra\ss e 85, D-22607 Hamburg, Germany}
\address{$^5$ Queen Mary University of London, Mile End Road, London E1 4NS, United Kingdom}
\address{$^6$ Karlsruher Institut f\"ur Technologie, Hermann-von-Helmholtz-Platz 1, D-76344 Eggenstein-Leopoldshafen, Germany}
\address{$^7$ Port d'Informaci\'o Cient\'ifica, Campus UAB, Edifici D, E-08193 Bellaterra, Spain}
\address{$^8$ Also Centro de Investigaciones Energ\'eticas, Medioambientales y Tecnol\'ogicas (CIEMAT), Madrid, Spain}
\address{$^9$ INFN, Sezione di Milano, via G. Celoria 16, I-20133 Milano, Italy}
\address{$^{10}$ Imperial College London, South Kensington Campus, London SW7 2AZ, United Kingdom}
\address{$^{11}$ CSC Tieteen Tietotekniikan Keskus Oy, P.O. Box 405, FI-02101 Espoo, Finland}
%\address{$^3$ Fermi National Accelerator Laboratory, Batavia, Il 60510, U.S.A.}
%\address{$^3$ Institute of High Energy Physics, 19B Yuquanlu, Shijingshan District, 100049 Beijing, China} 
%\address{$^{9}$ The University of Oxford, Denys Wilkinson Building, Keble Road, Oxford OX1 3RH, United Kingdom}
%\address{$^{12}$ California Institute of Technology, Pasadena, Ca 91125, U.S.A.}



\ead{david.kelsey@stfc.ac.uk, ipv6@hepix.org}

\begin{abstract}
IPv4 network addresses are running out and the deployment of IPv6 networking in many places is now well underway. Following the work of the HEPiX IPv6 Working Group, a growing number of sites in the Worldwide Large Hadron Collider Computing Grid (WLCG) are deploying dual-stack IPv6/IPv4 services. The aim of this is to support the use of IPv6-only clients, i.e. worker nodes, virtual machines or containers.
\par
The IPv6 networking protocols while they do contain features aimed at improving security also bring new challenges for operational IT security. 
%We have spent many decades understanding and fixing security problems and concerns in the IPv4 world. Many WLCG IT support teams have only just started to consider IPv6 security and they are far from ready to follow best practice, the guidance for which is not easy to distil. 
The lack of maturity of IPv6 implementations together with the increased complexity of some of the protocol standards 
%while noting that the new protocol stack allows for many of the same attack vectors as IPv4,
raise many new issues for operational security teams.
\par
The HEPiX IPv6 Working Group is producing guidance on best practices in this area. This paper considers some of the security concerns for WLCG in an IPv6 world and presents the HEPiX IPv6 working group guidance for the system administrators who manage IT services  on the WLCG distributed infrastructure, for their related site security and networking teams and for developers and software engineers working on WLCG applications.
\end{abstract}

\section{Introduction}
%section 1

Blah blah management board timeline.


\section{Some IPv6 security issues}
% This section presents some security concerns that arise from the use of IPv6 networking protocols. 



The introduction of any new protocol can of course introduce new security problems. For IPv6, these include the lack of maturity of the implementations which therefore are highly likely to still contain many new vulnerabilities. The need for IPv6-compliant monitoring and tools and the lack of education and experience of sysadmins both also cause concern. There are problems introduced by the need to support a transition process, including the complications arising from the use of dual-stack protocols and/or transition tunnels.

One of the potential advantages claimed by supporters of IPv6 was that security was explicitly addressed in the design. The mandatory use of IPsec (see RFC4301 \cite{rfc}), for example, seemed to be an early success but problems related to key management led to the mandatory support of IPsec being relaxed (see section 11 of RFC6434 \cite{rfc}). 

% One way of categorising the IPv6 security issues is as follows: those related to the new IPv6 packet extension headers, including fragmentation and other native header fields. Secondly there are security issues arising from the new Neighbor and Router Discovery protocols. Thirdly there are security issues relating to IPv4 to IPv6 transition tools, tunnelling and the new IPv6 support for mobility.

Examples of some other new features resulting in security concerns include:

\begin{enumerate}

\item {\bf Many more ICMP message types}. It is not possible to filter all of them, e.g. Path Maximumum Transmission Unit (MTU) Discovery has to work, but sites are recommended to filter some other types of ICMPv6. RFC4890 \cite{rfc} gives advice.

\item {\bf New methods for autoconfiguring addresses, and the locations of routers and DNS servers}. This is good news for the end-systems but networking teams must now protect against potential attacks, for example rogue Router Advertisements (see RFC6104 \cite{rfc}).

\item {\bf Longer IP addresses}. On first consideration, this appears to make brute force scans of end-systems much more difficult but that is not always true as there is often structure to the use of the address space. 

\item {\bf IPv6 does not allow packet fragmentation en-route}. The minimum MTU supported is 1280 bytes, but you can still hurt yourself by sending smaller fragments if you wish. 

\item Not really a feature of IPv6 proper, but much of the {\bf network stack and application code is not yet hardened} and therefore is potentially vulnerable to attack.

\item {\bf Transitional technologies} (e.g. tunnels) have intrinsic vulnerabilities but don't need to be there forever.

\end{enumerate}

The bad news is that many previous security issues from the IPv4-era have also not changed. 
%As long as all network monitoring and administration tools are up-to-date and aware of IPv6, 
Many earlier attacks are still possible:

\begin {itemize}
\item Broadcasts and Multicasts are still there, with a vengeance
\item Can still use IP headers for out-of-band communications
\item Can still pollute Ethernet address discovery (Neighbor Discovery instead of Address Resolution Protocol)
\item Can still run a rogue DHCP server
\item Can still try forging and injecting packets into the local network
\item The upper-layer protocols did not change
\end {itemize}






\section{Checklist for WLCG site system administrators and networking teams}





\begin{enumerate}


\item Make an addressing plan
One of the most important design decisions for a site networking team doing a deployment plan is to create a well thought out management plan for their IPv6 address space (linked to next topic too). Needs to include thoughts as to how to manage a dual-stack network. Address space (typically a /48 - default RFC3177) will have been allocated to the site by its NREN or other ISP. How many subnets? Routing architecture, address allocation within subnets etc. At the very least this should include .... See http://www.internetsociety.org/deploy360/resources/ipv6-address-planning-guidelines-for-ipv6-address-allocation/   and https://www.ripe.net/support/training/material/IPv6-for-LIRs-Training-Course/Preparing-an-IPv6-Addressing-Plan.pdf  


\item Decide whether to use DHCPv6 or SLAAC+DynDNS
The second most important decision (and very much linked to the one above) to be made by a site networking team is whether or not to use one of the important new features of IPv6, i.e. the end-system use of  IPv6 Stateless Address Autoconfiguration (see RFC4862).  And use of dynamic DNS.  Server systems may want to have fixed addresses (either manual or DHCPv6).



\item Ensure all security/network monitoring/logging are IPv6-capable
Important for networking teams, security teams and also end systems tools for sysadmins. All monitoring and logging tools (commercial, open-source, home written) need to be evaluated and tested for operation on IPv6. New longer addresses and multiple addresses per network card. Do they work in a dual-stack environment and can they simultaneously monitor both stacks.  What about tools analysing log files - does parsing work?


\item Filter IPv6 packets that enter and leave your network/system
IPv4-only networks will have end systems where IPv6 is enabled by default. May cause big security problems with IDS and Firewalls not handling IPv6 traffic correctly. If you don't want IPv6 best to turn it off and/or filter it at both the network and system level.

Filter packets with Extention headers.

For filtering of ICMPv6 packets - see next topic.


\item Filter ICMPv6 messages wisely
An important feature of IPv6. needed for path packet size determination and for ...
Some things can therefore not be blocked (unlike in IPv4).

\item Allow special-purpose headers only if needed
Extension Headers open up to all sorts of vulnerabilities - see RFC

\item Use synchronised IPv4/v6 access rules
For dual-stack networks MUCH much better to have identical firewall rules (site and end-system) for both stacks. Making them different can cause problems with



\item Deploy RA-Guard or otherwise deal with Rogue RAs
Neigbor Discovery and Router discovery are two important new features of IPv6. Opens up to several different security problems. Rogue routers can send out false router announcements (RAs) to persuade end systems to send packets to them for routing allowing for lack of privacy and man in the middle attacks.  Several ways of addressing this


\item Do not be tempted by transition technologies
By this  we mean think very carefully before deciding to use or allow the use of tunnelling technologies.  Dual-stack systems are the best approach. NAT64 is being used by some WLCG sites but we do not in general recommend this unless ...
problems with tunnels and protocol translations are ...

\item Filter/disable IPv6-on-IPv4 tunnels
We recommend not using such tunnels (see above).  So we suggest that sites should filter or disable these - done as follows?


\end{enumerate}

From ISGC2016 talk:

Control IPv6 if not using it
Use Dual-stack and avoid use of tunnels wherever possible
Drop packets containing RH Type 0 and unknown option headers
Deny packets that do not follow rules for extension headers
Filter IPv6 packets that enter and leave your network
Restrict who can send messages to multicast group addresses
Create an Address management plan
Create a Security Policy for IPv6 (same as IPv4)
Block unnecessary ICMPv6
Protect against LAN RA, ND and DHCP attacks
NDPMON and RAFIXD on critical segments
Check/modify all security monitoring, logging and parsing tools





\section{Checklist for developers}
\begin{figure}
\begin{framed}
{\tt\small
struct hostent  *resolved\_name=NULL;\\
struct servent  *resolved\_serv=NULL;\\
struct protoent *resolved\_proto=NULL;\\
static char     *dest\_host="some.ip.host", *dest\_serv="ipservice";\\
struct sockaddr\_in destination;\\ 
resolved\_host = gethostbyname(dest\_host);\\
resolved\_serv = getservbyname(dest\_serv, NULL);\\
if (resolved\_host != NULL \&\& resolved\_serv != NULL) \{\\
\qquad destination.sin\_family = resolved\_name->h\_addrtype;\\
\qquad destination.sin\_port = htons(resolved\_serv->s\_port);\\
\qquad memcpy(\&destination.sin\_addr, resolved\_host->h\_addr\_list[0],\\
\qquad\qquad resolved\_host->h\_length);\\
\qquad resolved\_proto = getprotobyname(resolved\_serv->s\_proto)\\
\qquad if (resolved\_proto != NULL) \{\\
\qquad\qquad int fd = socket(AF\_INET, SOCK\_STREAM, resolved\_proto->p\_proto);\\
\qquad\qquad connect(fd, \&destination, sizeof(destination));\\
\qquad\qquad /* Check for errors, connect, etc... */\\
\qquad \}\\
\}
}
\end{framed}
\par
\begin{framed}
{\tt\small
struct addrinfo ai\_req, *ai\_ans, *cur\_ans;\\
static char     *dest\_host="some.ip.host", *dest\_serv="ipservice";\\
ai\_req.ai\_flags = 0;\\
ai\_req.ai\_family = PF\_UNSPEC;\\
ai\_req.ai\_socktype = SOCK\_STREAM;\\
ai\_req.ai\_protocol = 0; /* Any protocol is OK */\\
if (getaddrinfo(dest\_host, dest\_serv, \&ai\_req, \&ai\_ans) != 0) \{\\
\qquad for (cur\_ans = ai\_ans; cur\_ans != NULL; cur\_ans = cur\_ans->ai\_next) \{\\
\qquad\qquad int fd = socket(cur\_ans->ai\_family, cur\_ans->ai\_socktype,\\
\qquad\qquad\qquad cur\_ans->ai\_protocol);\\
\qquad\qquad connect(fd\_socket,cur\_ans->ai\_addr,cur\_ans->ai\_addrlen);\\
\qquad\qquad /* Check for errors - This loop has the ability to change the */\\
\qquad\qquad /* order of the getaddrinfo results! */\\
\qquad \}\\
\}
}
\end{framed}
\caption{C code snippets showing how the basic IP service resolution and connection 
 changes from legacy IPv4-only to a dual-stack or IPv6-only environment. This represents
 the zeroth-order porting effort for much IPv4-only code. The newer structure
 is more terse, but the changes are extensive enough, both syntactically and
 semantically, to probably trigger the refactoring of much larger sections of code.}
\label{fig:pseudocode}
\end{figure}
When applications developed in the golden
era of IPv4-only Internet face the transition to IPv6, the brunt of the
work often falls on the shoulders of developers, who belong to a different
generation from the original authors. Figure \ref{fig:pseudocode} tries to
visualise the extent of the changes that the core code of any
IP-capable application undergoes in the transition. In addition to
the extensively different syntax, there is a
fundamental $1\rightarrow N$ change here:
no IP endpoint can be satisfied with handling just {\it one}
IP address (as any public IPv6 endpoint communicates via the
public and on the link-local address at least), but loops and address
ordering start appearing everywhere. We identify the following implications of this
fundamental fact on developers' practice, in rough descending order
of importance:
\begin{enumerate}
\item {\bf Plan for extensive testing.}\\
      The {\it syntactic} change of the core IP networking code in the
      IPv4$\rightarrow$IPv6 transition is large enough to oftentime
      justify the refactoring of larger portions of code. The 
      {\it semantic} $1\rightarrow N$ change may be {\it forcing}
      some rethinking at the design level. A possible temptation here
      is to provide parallel sections of code that handle the IPv6 case
      only: a few other reasons why this may not be a good idea are
      listed below. In any case, there is an implicit expectation that 
      a change that should be affecting the {\it transport} layer of
      the network only should cause no ripple in the upper layers, i.e.
      that the perceived responsiveness, performance and reliability
      of the code remain unchanged. {\it Extensive} stress-testing should
      therefore be planned on IPv6-ported code. 
\item {\bf Respect the sysadmin protocol preferences.}\\
      Code that binds and connects IP sockets is suddenly faced with 
      making choices that used to be delegated to the operating system or
      networking-capable libraries. Lists of addresses may be received in
      a given order, but it's now the responsibility of the socket-handling
      code to iterate and re-iterate on the list, handle exceptions and
      possibly operate in parallel on various entries
      to implement some form of `happy-eyeballs'\footnote{See RFC6555 \cite{rfc}.}
      alghorithm. As the ordering of both source and destination addresses
      established at the system level by the system
      administrator\footnote{Via {\tt /etc/gai.conf}, {\tt ip addrlabel} or their equivalent.}
      may have security implications, developers should go the extra mile to 
      keep that ordering even if they have to reshuffle the list for any reason.
      Applications should allow users to prefer/enable either IPv4 or IPv6 via
      configuration, but should always honor the system-level administrator's choice 
      by default.
\item {\bf Port all existing security measures.}\\
      Fresh new code that hasn't been tested broadly and in the 
      wild is {\it per se} attractive to anyone looking for
      malicious exploits. Especially in the case where IPv6-specific
      code or processes are developed for {\it parallel} deployment with
      well-proven IPv4 code, one should make sure that any security measure,
      filter or wisdom that was included in the code for the IPv4 case isn't
      simply forgotten for IPv6. While it may not be immediately apparent,
      {\it all} constructs that are meaningful for IPv4 have their
      translation or counterpart for IPv6.
\end{enumerate}


\section{Summary}
% Summary

In this paper, we have discussed some of the security concerns that arise from the deployment of the IPv6 networking protocols. In particular, we have presented our IPv6 security checklist for sysadmins and site networking/security teams at WLCG sites. We have also presented a checklist for WLCG/HEP application developers and software engineers. We welcome feedback from sites and
developers on the contents of these lists according to their experiences during the transition.  HEP IT support staff have spent many decades understanding and fixing security problems and concerns in the IPv4 world. We confidently predict that securing IPv6 will take an equally long time!

\section*{Acknowledgements}
% Acknowledgements

The authors acknowledge the contributions to this work made by former members of the HEPiX IPv6 Working Group and other colleagues within WLCG. In particular we express our thanks to Raul Lopes, Simon Furber and colleagues at Brunel University London for their extensive testing of IPv6-only compute nodes. We also thank Edgar Fajardo Hernandez from the University of California, San Diego for his work on the validation of IPv6 and glideinWMS for the CMS experiment.

The authors also acknowledge the support and collaboration of many other colleagues in their respective institutes, experiments and IT Infrastructures, together with the funding received by these from many different sources. 

These include but are not limited to the following:

The Worldwide LHC Computing Grid (WLCG) project is a global collaboration of more than 170 computing centres in 42 countries, linking up national and international grid infrastructures. Funding is acknowledged from many national funding bodies and we acknowledge the support of several operational infrastructures including EGI, OSG and NDGF/NeIC.

EGI acknowledges the funding and support received from the European Commission and the many National Grid Initiatives and other members. The EGI-Engage project is co-funded by the European Commission (grant number 654142).

Authors from the UK acknowledge the funding and support received from the Science and Technology Facilities Council via the GridPP project.





% Add other sections as appropriate...

%\par
\pagebreak
\section*{References}

\begin{thebibliography}{1}
\bibitem{ipv6wg} {\tt http://hepix-ipv6.web.cern.ch}
\bibitem{rfc} All Internet Engineering Task Force Requests For Comments (RFC) documents are available
from URLs such as http://www.ietf.org/rfc/rfcNNNN.txt where NNNN is the RFC number, for example {\tt http://www.ietf.org/rfc/rfc2460.txt}
\bibitem{planningguides} There is abundant refernce material at
{\tt http://www.internetsociety.org/deploy360/resources/ipv6-address-planning-guidelines-for-ipv6-address-allocation/} and {\tt https://www.ripe.net/support/training/material/IPv6-for-LIRs-Training-Course/Preparing-an-IPv6-Addressing-Plan.pdf}.
\bibitem{CiscoBook}
    Hogg S, Vyncke E - IPv6 Security, Cisco Press 2009, ISBN-13: 978-1-58705-594-2
\end{thebibliography}


\end{document}

{

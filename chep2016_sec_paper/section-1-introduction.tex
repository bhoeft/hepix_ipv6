%section 1

The much-heralded exhaustion of the IPv4 networking address space is with us. The HEPiX
IPv6 Working Group \cite{ipv6wg} has been investigating the many issues feeding into the deployment of IPv6 in HEP in general and more specifically in WLCG. The group's paper at CHEP2015 \cite{ipv6chep2015} presented the testing and deployment of dual-stack data storage services with the aim of soon being able to support the use of IPv6-only CPU. Since then, WLCG has approved our plan to support such use of IPv6-only CPU from April 2017 \cite{ipv6chep2016}.

One of the important concerns for this migration to IPv6 relates to operational security. The IPv6 networking protocols while they do contain features aimed at improving security also bring new challenges. We have spent many decades understanding and fixing security problems and concerns in the IPv4 world. Many WLCG site support teams have only just started to consider IPv6 security and they are far from ready to be able to follow best practice. 

There is much information available on IPv6 security but the fact that there are so many documents on the topic does not make it easy for WLCG system administrators, hereafter abbreviated to ``sysadmins'', to digest and identify the key issues. This paper is not competing with the other information but gives pointers to these other books and papers. The IPv6 working group has decided to produce and maintain two short checklists of the key IPv6 security issues to be addressed as a starting point for WLCG sysadmins, for their site networking and security teams and also for WLCG/HEP application developers and software engineers. This is based on the experience of those sites active in the HEPiX IPv6 working group.

We have found the following to be useful sources of fuller information on IPv6 Security. These contain much more background information and fuller exploration of the details.

\begin {enumerate}
\item A large and complete text-book from Cisco Press studying the whole subject matter \cite{CiscoBook}.
\item NIST SP 800-119, Guidelines for the Secure Deployment of IPv6. This is a shorter but still complete study of the topic \cite{nist800-119}.
\item ``10 myths of IPv6 security from the Internet society''; all interesting and amusing! \cite{10myths}.
\item A SANS white paper on IPv6 attack and defense \cite {SANSipv6}.
\item An ERNW hardening guide for Linux servers; the best guidance we have found so far on IPv6 and Linux Systems \cite {ERNWipv6}.
\item There are many IETF RFC documents on IPv6 and IPv6 security. In particular, we recommend ``Enterprise IPv6 Deployment Guidelines'', RFC7381 \cite{rfc}.
\end {enumerate}

The checklists are the IPv6 working group's current list of issues to be considered. The checklists are maintained on the group web site \cite {ipv6wg}. We welcome feedback from sites and developers on the contents of these lists according to their experiences during the transition. Updates and additions will be made as required.

This paper is organised as follows. Section 2 presents a brief introduction to some of the potential vulnerabilities and concerns related to IPv6. Section 3 contains the checklist for sysadmins and site networking/security teams. Section 4 presents our checklist aimed at application developers. The paper ends with a short summary in Section 5.






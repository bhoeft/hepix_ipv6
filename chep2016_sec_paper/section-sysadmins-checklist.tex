(This Francesco's text - modify this!)
When applications developed in the golden
era of IPv4-only Internet face the transition to IPv6 the brunt of the
work often falls on the shoulders of developers, who often belong to a different
generation as the original authors. Figure \ref{fig:pseudocode} tries to
visualise the {\it extent} of the changes that the core code of any
IP-capable application undergoes in the transition. In addition to
the extensively different syntax, there is a
fundamental $1\rightarrow N$ change here:
no IP endpoint can be satisfied with handling just {\it one}
IP address (as any public IPv6 endpoint communicates via the
public and on the link-local address at least), but loops and address
ordering start appearing everywhere. We identify the following implications of this
fundamental fact on developers' practice, in rough descending order
of importance:
\begin{enumerate}
\item The {\it syntactic} change of the core IP networking code in the
      IPv4$\rightarrow$IPv6 transition is large enough to oftentime
      justify the refactoring of larger portions of code. The 
      {\it semantic} $1\rightarrow N$ change may be {\it forcing}
      some rethiking at the design level. A possible temptation here
      is to provide parallel sections of code that handle the IPv6 case
      only: a few other reasons why this may not be a good idea are
      listed below. In any case, there is an implicit expectation that 
      a change that should be affecting the {\it transport} layer of
      the network only should cause no ripple in the upper layers, i.e.
      that the perceived responsiveness, performance and reliability
      of the code remain unchanged. {\it Extensive} stress-testing should
      therefore be planned on IPv6-ported code. 
\item Code that binds and connects IP sockets is suddenly faced with 
      making choices that used to be delegated to the operating system or
      networking-capable libraries. Lists of addresses may be received in
      a given order, but it's now the responsibility of the socket-handling
      code to iterate and re-iterate on the list, handle exceptions and
      possibly operate in parallel on various entries
      to implement some form of 'happy-eyeballs'\footnote{See RFC6555 \cite{rfc}.}
      alghorithm. As the ordering of both source and destination addresses
      established at the system level by the system
      administrator\footnote{Via {\tt /etc/gai.conf}, {\tt ip addrlabel} or their equivalent.}
      may have security implications, developers should go the extra mile to 
      keep that ordering even if they have to reshuffle the list for any reason.
\item Fresh new code that hasn't been tested broadly and in the 
      wild is {\it per se} attractive to anyone looking for
      malicious exploits. Especially in the case where IPv6-specific
      code or processes are developed for {\it parallel} deployment with
      well-proven IPv4 code, one should make sure that any security measure,
      filter or wisdom that was included in the code for the IPv4 case isn't
      simply forgotten for IPv6. While it may not be immediately apparent,
      {\it all} constructs that are meaningful for IPv4 have their
      translation or counterpart for IPv6.
\end{enumerate}

IPv6 Security issues

Paper
starting with extension headers, fragmentation and other native header fields. Subsequently, Neighbor and Multicast Listener Discovery are discussed, followed by tunneling and mobility support.

New features (from the poster)

Many more ICMP message types! Cannot filter all of them (MTU discovery has to work). Must filter some of them. RFC4890 gives advice

New methods for autoconfiguring addresses, routes, DNS. Good for the end-user. Must do something against rogue Router Advertisements (see RFC6104)

Longer IP addresses. Hey, everyone knows that. They may slow down brute force scans. But no bad guy is that crude...

Cannot fragment packets en-route.  Minimum MTU: 1280. But you can still hurt yourself and send small fragments if you wish. Some good news, at least

Not really a feature of IPv6 proper, but much of the network stack and application code is enticingly fresh!

Transitional technologies (e.g. tunnels) have intrinsic vulnerabilities but don't need to be there forever...


Business as usual (from the poster)

As long as all network monitoring and administration tools are up-to-date and (therefore) aware of IPv6.

Broadcasts and Multicasts are still there, with a vengeance.
Can still use IP headers for out-of-band communications.
Can still pollute Ethernet address discovery (ND instead of ARP).
Can still run a rogue DHCP server.
Can still try forging and injecting packets into the local network. 
Upper-layer protocols did not change!


Advantages of a new design
Security: important part of the IPv6 initial design
Down-sides
Lack of maturity
New vulnerabilities and attack vectors
Need IPv6-compliant monitoring and tools
Lack of education and experience
Problems of transition ? dual-stack, tunnels
BUT - Many threats/attacks happen at layers above/below the network layer
And are therefore exactly the same as in IPv4
Malware, phishing, buffer overflows, cross-site scripting, DDoS etc etc

% This section presents some security concerns that arise from the use of IPv6 networking protocols. 



The introduction of any new protocol can of course introduce new security problems. For IPv6, these include the lack of maturity of the implementations which therefore are highly likely to still contain many new vulnerabilities. The need for IPv6-compliant monitoring and tools and the lack of education and experience of sysadmins both also cause concern. There are problems introduced by the need to support a transition process, including the complications arising from the use of dual-stack protocols and/or transition tunnels.

One of the potential advantages claimed by supporters of IPv6 was that security was explicitly addressed in the design. The mandatory use of IPsec (see RFC4301 \cite{rfc}), for example, seemed to be an early success but problems related to key management led to the mandatory support of IPsec being relaxed (see section 11 of RFC6434 \cite{rfc}). 

% One way of categorising the IPv6 security issues is as follows: those related to the new IPv6 packet extension headers, including fragmentation and other native header fields. Secondly there are security issues arising from the new Neighbor and Router Discovery protocols. Thirdly there are security issues relating to IPv4 to IPv6 transition tools, tunnelling and the new IPv6 support for mobility.

Examples of some other new features resulting in security concerns include:

\begin{enumerate}

\item {\bf Many more ICMP message types}. It is not possible to filter all of them, e.g. Path Maximumum Transmission Unit (MTU) Discovery has to work, but sites are recommended to filter some other types of ICMPv6. RFC4890 \cite{rfc} gives advice.

\item {\bf New methods for autoconfiguring addresses, and the locations of routers and DNS servers}. This is good news for the end-systems but networking teams must now protect against potential attacks, for example rogue Router Advertisements (see RFC6104 \cite{rfc}).

\item {\bf Longer IP addresses}. On first consideration, this appears to make brute force scans of end-systems much more difficult but that is not always true as there is often structure to the use of the address space. 

\item {\bf IPv6 does not allow packet fragmentation en-route}. The minimum MTU supported is 1280 bytes, but you can still hurt yourself by sending smaller fragments if you wish. 

\item Not really a feature of IPv6 proper, but much of the {\bf network stack and application code is not yet hardened} and therefore is potentially vulnerable to attack.

\item {\bf Transitional technologies} (e.g. tunnels) have intrinsic vulnerabilities but don't need to be there forever.

\end{enumerate}

The bad news is that many previous security issues from the IPv4-era have also not changed. 
%As long as all network monitoring and administration tools are up-to-date and aware of IPv6, 
Many earlier attacks are still possible:

\begin {itemize}
\item Broadcasts and Multicasts are still there, with a vengeance
\item Can still use IP headers for out-of-band communications
\item Can still pollute Ethernet address discovery (Neighbor Discovery instead of Address Resolution Protocol)
\item Can still run a rogue DHCP server
\item Can still try forging and injecting packets into the local network
\item The upper-layer protocols did not change
\end {itemize}





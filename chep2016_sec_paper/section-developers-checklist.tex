\begin{figure}
\begin{framed}
{\tt\small
struct hostent  *resolved\_name=NULL;\\
struct servent  *resolved\_serv=NULL;\\
struct protoent *resolved\_proto=NULL;\\
static char     *dest\_host="some.ip.host", *dest\_serv="ipservice";\\
struct sockaddr\_in destination;\\ 
resolved\_host = gethostbyname(dest\_host);\\
resolved\_serv = getservbyname(dest\_serv, NULL);\\
if (resolved\_host != NULL \&\& resolved\_serv != NULL) \{\\
\qquad destination.sin\_family = resolved\_name->h\_addrtype;\\
\qquad destination.sin\_port = htons(resolved\_serv->s\_port);\\
\qquad memcpy(\&destination.sin\_addr, resolved\_host->h\_addr\_list[0],\\
\qquad\qquad resolved\_host->h\_length);\\
\qquad resolved\_proto = getprotobyname(resolved\_serv->s\_proto)\\
\qquad if (resolved\_proto != NULL) \{\\
\qquad\qquad int fd = socket(AF\_INET, SOCK\_STREAM, resolved\_proto->p\_proto);\\
\qquad\qquad connect(fd, \&destination, sizeof(destination));\\
\qquad\qquad /* Check for errors, connect, etc... */\\
\qquad \}\\
\}
}
\end{framed}
\par
\begin{framed}
{\tt\small
struct addrinfo ai\_req, *ai\_ans, *cur\_ans;\\
static char     *dest\_host="some.ip.host", *dest\_serv="ipservice";\\
ai\_req.ai\_flags = 0;\\
ai\_req.ai\_family = PF\_UNSPEC;\\
ai\_req.ai\_socktype = SOCK\_STREAM;\\
ai\_req.ai\_protocol = 0; /* Any protocol is OK */\\
if (getaddrinfo(dest\_host, dest\_serv, \&ai\_req, \&ai\_ans) != 0) \{\\
\qquad for (cur\_ans = ai\_ans; cur\_ans != NULL; cur\_ans = cur\_ans->ai\_next) \{\\
\qquad\qquad int fd = socket(cur\_ans->ai\_family, cur\_ans->ai\_socktype,\\
\qquad\qquad\qquad cur\_ans->ai\_protocol);\\
\qquad\qquad connect(fd\_socket,cur\_ans->ai\_addr,cur\_ans->ai\_addrlen);\\
\qquad\qquad /* Check for errors - This loop has the ability to change the */\\
\qquad\qquad /* order of the getaddrinfo results! */\\
\qquad \}\\
\}
}
\end{framed}
\caption{C code snippets showing how the basic IP service resolution and connection 
 changes from legacy IPv4-only to a dual-stack or IPv6-only environment. This represents
 the zeroth-order porting effort for much IPv4-only code. The newer structure
 is more terse, but the changes are extensive enough, both syntactically and
 semantically, to probably trigger the refactoring of much larger sections of code.}
\label{fig:pseudocode}
\end{figure}

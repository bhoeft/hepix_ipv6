\begin{figure}
\begin{framed}
{\tt\small
struct hostent  *resolved\_name=NULL;\\
struct servent  *resolved\_serv=NULL;\\
struct protoent *resolved\_proto=NULL;\\
static char     *dest\_host="some.ip.host", *dest\_serv="ipservice";\\
struct sockaddr\_in destination;\\ 
resolved\_host = gethostbyname(dest\_host);\\
resolved\_serv = getservbyname(dest\_serv, NULL);\\
if (resolved\_host != NULL \&\& resolved\_serv != NULL) \{\\
\qquad destination.sin\_family = resolved\_name->h\_addrtype;\\
\qquad destination.sin\_port = htons(resolved\_serv->s\_port);\\
\qquad memcpy(\&destination.sin\_addr, resolved\_host->h\_addr\_list[0],\\
\qquad\qquad resolved\_host->h\_length);\\
\qquad resolved\_proto = getprotobyname(resolved\_serv->s\_proto)\\
\qquad if (resolved\_proto != NULL) \{\\
\qquad\qquad int fd = socket(AF\_INET, SOCK\_STREAM, resolved\_proto->p\_proto);\\
\qquad\qquad connect(fd, \&destination, sizeof(destination));\\
\qquad\qquad /* Check for errors, connect, etc... */\\
\qquad \}\\
\}
}
\end{framed}
\par
\begin{framed}
{\tt\small
struct addrinfo ai\_req, *ai\_ans, *cur\_ans;\\
static char     *dest\_host="some.ip.host", *dest\_serv="ipservice";\\
ai\_req.ai\_flags = 0;\\
ai\_req.ai\_family = PF\_UNSPEC;\\
ai\_req.ai\_socktype = SOCK\_STREAM;\\
ai\_req.ai\_protocol = 0; /* Any protocol is OK */\\
if (getaddrinfo(dest\_host, dest\_serv, \&ai\_req, \&ai\_ans) != 0) \{\\
\qquad for (cur\_ans = ai\_ans; cur\_ans != NULL; cur\_ans = cur\_ans->ai\_next) \{\\
\qquad\qquad int fd = socket(cur\_ans->ai\_family, cur\_ans->ai\_socktype,\\
\qquad\qquad\qquad cur\_ans->ai\_protocol);\\
\qquad\qquad connect(fd\_socket,cur\_ans->ai\_addr,cur\_ans->ai\_addrlen);\\
\qquad\qquad /* Check for errors - This loop has the ability to change the */\\
\qquad\qquad /* order of the getaddrinfo results! */\\
\qquad \}\\
\}
}
\end{framed}
\caption{C code snippets showing how the basic IP service resolution and connection 
 changes from legacy IPv4-only to a dual-stack or IPv6-only environment. This represents
 the zeroth-order porting effort for much IPv4-only code. The newer structure
 is more terse, but the changes are extensive enough, both syntactically and
 semantically, to probably trigger the refactoring of much larger sections of code.}
\label{fig:pseudocode}
\end{figure}
When applications developed in the golden
era of IPv4-only Internet face the transition to IPv6, the brunt of the
work often falls on the shoulders of developers who belong to a different
generation from the original authors. Figure \ref{fig:pseudocode} tries to
visualise the extent of the changes that the core code of any
IP-capable application undergoes in the transition. In addition to
the extensively different syntax, there is a
fundamental $1\rightarrow N$ change here:
no IP endpoint can be satisfied with handling just {\it one}
IP address (as any public IPv6 endpoint communicates via the
public and on the link-local address at least), but loops and address
ordering start appearing everywhere. We identify the following implications of this
fundamental fact on developers' practice, in roughly descending order
of importance:
\begin{enumerate}
\item {\bf Plan for extensive testing.}\\
      The {\it syntactic} change of the core IP networking code in the
      IPv4$\rightarrow$IPv6 transition is large enough to oftentime
      justify the refactoring of larger portions of code. The 
      {\it semantic} $1\rightarrow N$ change may be {\it forcing}
      some rethinking at the design level. A possible temptation here
      is to provide parallel sections of code that handle the IPv6 case
      only: a few other reasons why this may not be a good idea are
      listed below. In any case, there is an implicit expectation that 
      a change that should be affecting the {\it transport} layer of
      the network only should cause no ripple in the upper layers, i.e.
      that the perceived responsiveness, performance and reliability
      of the code remain unchanged. {\it Extensive} stress-testing should
      therefore be planned on IPv6-ported code. 
\item {\bf Respect the sysadmin protocol preferences.}\\
      Code that binds and connects IP sockets is suddenly faced with 
      making choices that used to be delegated to the operating system or
      networking-capable libraries. Lists of addresses may be received in
      a given order, but it's now the responsibility of the socket-handling
      code to iterate and re-iterate on the list, handle exceptions and
      possibly operate in parallel on various entries
      to implement some form of `happy-eyeballs'\footnote{See RFC6555 \cite{rfc}.}
      algorithm. As the ordering of both source and destination addresses
      established at the system level by the system
      administrator\footnote{Via {\tt /etc/gai.conf}, {\tt ip addrlabel} or their equivalent.}
      may have security implications, developers should go the extra mile to 
      keep that ordering even if they have to reshuffle the list for any reason.
      Applications should allow users to prefer/enable either IPv4 or IPv6 via
      configuration, but should always honor the system-level administrator's choice 
      by default.
\item {\bf Port all existing security measures.}\\
      Fresh new code that hasn't been tested broadly and in the 
      wild is {\it per se} attractive to anyone looking for
      malicious exploits. Especially in the case where IPv6-specific
      code or processes are developed for {\it parallel} deployment with
      well-proven IPv4 code, one should make sure that any security measure,
      filter or wisdom that was included in the code for the IPv4 case isn't
      simply forgotten for IPv6. While it may not be immediately apparent,
      {\it all} constructs that are meaningful for IPv4 have their
      translation or counterpart for IPv6.
\end{enumerate}
